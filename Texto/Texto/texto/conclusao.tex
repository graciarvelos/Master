\chapter[Conclusões]{Conclusões \label{cap:conclusao}}
%\addcontentsline{toc}{chapter}{Conclusão}

Nesse trabalho realizamos a descrição atomística da interface água/metal sem a presença de perturbações externas e considerando a aplicação de um potencial externo. Isso foi feito através da adsorção de um monômero e uma camada de água em superfícies metálicas. Para aplicar a diferença de potencial foi utilizado formalismo de NEGF+DFT e o efeito do potencial externo foi mensurado através das propriedades estruturais, eletrônicas e vibracionais. 

Primeiramente, caracterizamos a interface água/metal sem a presença de perturbações externas. A partir da análise do monômero adsorvido no Pd(111), encontramos que a orientação mais estável é a orientação \textit{flat}, seguida da orientação \textit{up}. Além disso, obtivemos que as forças de dispersão de van der Waals são relevantes na descrição das interações água/metal e afetam as energias de adsorção e as propriedades vibracionais. A acurácia e precisão desses resultados foram fundamentadas pelos testes de convergência de pontos \textit{k} e tamanho de superfície.

Em seguida, aumentamos a complexidade do sistema e realizamos simulações com três tipos de camadas bidimensionais: H-Down, H-Up e H-Down/Up. Assim, foi possível caracterizar a competição entre as ligações de hidrogênio e interações água/metal. Essa caracterização mostrou que a camada H-Down é a mais estável devido à presença das interações entre as moléculas de água \textit{flat-down} e o metal.

Após descrever a interface água/Pd(111) sem a presença de perturbações externas, analisamos como a aplicação do potencial externo afeta o comportamento dessa interface e também do monômero adsorvido no Au(111). De modo geral, observamos que para potenciais positivos a molécula tende a se afastar do eletrodo metálico, ao passo que para potenciais negativos a molécula tende a se aproximar da superfície carregada. Como resultado, as frequências de \textit{stretching} diminuem para potenciais positivos e vice-versa. Assim, a posição de mínima energia do monômero é resultado do efeito eletrostático somado ao efeito do potencial sobre a repulsão de Pauli. 

Além disso, observamos que a reatividade do metal afeta fortemente o comportamento da molécula de água na superfície carregada. Para tanto, observamos que para um mesmo valor de potencial, as distâncias $ d_{OM} $ e a inclinação $ \alpha $ do monômero adsorvido no Au(111) foram superiores às do Pd(111). Consequentemente, as frequências vibracionais do Pd(111) foram menores. Além disso, observamos que o potencial externo afetou o sítio preferencial de adsorção do monômero nos dois metais.

Uma vez caracterizado o efeito do potencial externo sobre a interação água/metal, investigamos como o potencial afeta ligações de hidrogênio. Para isso, utilizamos a camada H-Down adsorvida no Au(111) e observamos que o potencial externo afera principalmente o comportamento estrutural das moléculas \textit{flat}. Assim, as moléculas \textit{flat} reproduzem o comportamento observado no monômero mas com menor intensidade.


Em relação às propriedades vibracionais, observa-se que as frequências inferiores descrevem as ligações O-H das moléculas \textit{flat-down} que participam das ligações de hidrogênio e as frequências superiores correspondem às ligações O-H das moléculas \textit{flat}. Além disso, observamos que potenciais negativos intensificam as ligações de hidrogênio e potenciais positivos enfraquecem as interações água/metal e ligações de hidrogênio.

Através do cálculo de forças fora do equilíbrio, conseguimos realizar a otimização das coordenadas dos sistemas água/metal e obter as propriedades estruturais, eletrônicas e vibracionais. Assim, como perspetiva futura é importante analisar o comportamento das estruturas ao aplicar potenciais maiores, bem como analisar o efeito do potencial sobre estruturas maiores de água. Além disso, como esse formalismo permite o cálculos de forças fora do equilíbrio, é possível, em princípio, estender esse formalismo a simulações de dinâmica molecular \textit{ab initio}, bem como incluir o efeito de solventes através de modelos contínuos.