\begin{agradecimentos}
Gostaria de agradecer, primeiramente, à duas pessoas que partiram, mas que honraram os anos aqui vividos amando aos seus intensamente: meu pai e Telma Cristiane. Meu pai, além do exemplo de integridade deixado como legado, foi o primeiro a despertar em mim a paixão pela ciência. Telma, com sua inteligência e irreverência marcou o coração e a vida de cada pessoa que a conheceu e assim, não somente marcou a minha vida, mas foi uma das bases que tornou possível esse momento. Sou imensamente grata por ter tido o privilégio de conviver e por ter sido amada profundamente por vocês. 

Agradeço à minha mãe pelo apoio incondicional e por ser esse porto seguro para nossa família. Aos meus irmãos Gustavo, Gabriel e Geovanna agradeço por terem me apoiado ao longo dessa jornada. Às minhas irmãs de coração Juliany, Sannya, Simone e Gabi: vocês personificam a cumplicidade, a parceria e a sinceridade que só se encontram numa amizade. Ao meu melhor amigo, companheiro e namorado Lucas, agradeço por todo suporte e por tirar minhas dúvidas de matemática. 

Agradeço aos meus avós de sangue e de coração: Iraci, Raymundo, Terezinha, Dinha, Dona Lica e Eunice. Vocês, com uma imensa sabedoria adquirida pelo passar dos anos, me revelam tesouros que só o tempo pode fornecer. Às minhas madrinhas Edevânia, Jovenilce e Josa, que sempre têm um sorriso caloroso e um abraço aconchegante a oferecer; aos meus tios Solimar, Junior e Ênio e às minhas tias Elaine, Cilene, Augusta Cássia, Ivonete Jesus por serem uma fonte de inspiração, força e perseverança. Aos meus primos Alynne, Priscyla, Mariane, Luís Eduardo, Carol, Douglas, João Victor, Anthônio, João Miguel e Emily pelo apoio e carinho. À Nicolly Eduarda por nos ensinar a acreditar. Às minhas amigas Iany, Cris e Érika por todos os conselhos e momentos compartilhados.

Sou profundamente grata pela experiência de ter sido aluna do Programa de Pós Graduação em Nanociências e em Materiais Avançados da UFABC. Agradeço em especial à minha orientadora Luana Pedroza pela paciência, dedicação, confiança e principalmente por ser uma inspiração e exemplo de pesquisadora. Ademais, agradeço aos professores Gustavo Dalpian, Celso Nishi, Alexandre Rocha e Antonio Pedroza pelo apoio e discussões ao longo desse trabalho.

Ao longo da minha jornada tive o privilégio de conviver com pessoas que admiro e que tornaram a caminhada mais leve. Agradeço aos meus amigos da UnB Ana Caroline e João Augusto, que mesmo longe foram fundamentais nessa etapa. Agradeço aos meus amigos da UFABC: Camila, Eduardo, Elton, Leo, Reinaldo, João e Café pelo companheirismo e por dividirem comigo as alegrias e desafios desse período na UFABC. 

O presente trabalho foi realizado com apoio da Coordenação de Aperfeiçoamento de Pessoal de Nível Superior - Brasil (CAPES) - Código de Financiamento 001 e pela Fundação de Amparo à Pesquisa do Estado de São Paulo (FAPESP) através da concessão de bolsa de estudo de Mestrado (processo nº 2020/16593-4) vinculado ao projeto Jovem Pesquisador nº 2017/10292-0. As opiniões, hipóteses e conclusões ou recomendações expressas neste material são de responsabilidade do(s) autor(es) e não necessariamente refletem a visão da FAPESP. 
\end{agradecimentos}