\begin{resumo}
 
%Contexto breve e problema investigado e não resolvido. Duas a três sentenças são suficientes

%Método breve e resultado mais importante

%Conclusões e implicações

Novas maneiras de geração de energia limpa estão relacionadas à reações que ocorrem em interfaces água/metal, por exemplo via processos eletroquímicos e de catálise heterogênea. Assim, a caracterização em nível atômico desses processos é fundamental para o aprimoramento das atuais e o desenvolvimento de novas fontes de energia renováveis. Para isso, é necessário compreender a estrutura e as propriedades eletrônicas de interfaces água/metal, bem como as respostas do sistema à aplicação de uma diferença de potencial externa. Para tanto, neste trabalho utilizou-se a Teoria do Funcional da Densidade (DFT) para caracterizar as interações de água (monômero e camada) em superfícies metálicas (Paládio e Ouro). Também utilizou-se o formalismo de Funções de Green Fora do Equilíbrio (NEGF) acoplado ao DFT para investigar o efeito da aplicação de um potencial externo sobre a interface água/metal e investigar como as propriedades estruturais, eletrônicas e vibracionais são alteradas. Assim, observamos que as moléculas tendem a se aproximarem do metal quando esse está carregado negativamente e se afastarem ao carregar positivamente. Essas alterações afetam as propriedades vibracionais e dependem fortemente da reatividade do metal.


%Aumentar a eficiência da conversão energética de células solares está ligado à melhoria dos processos eletroquímicos e de catálise heterogênea. Assim, a caracterização a nível atômico desses processos é fundamental para o aprimoramento das atuais e o desenvolvimento de novas fontes de energia renováveis. Para isso é necessário compreender a dinâmica e as propriedades eletrônicas da interface água/metal, bem como as respostas do sistema a pertubações externas. Para tanto, utilizou-se a Teoria do Funcional da Densidade (DFT) para caracterizar as interações provenientes do processo de adsorção da interface água/paládio, no qual foi considerado os efeitos das forças de dispersão através do funcional VDW-BH. Por meio da adsorção do monômero foi estudado como esse funcional fornece mais detalhes sobre o sistema e por meio da adsorção de uma camada de água observou-se que a interação água/metal melhora a ligação de hidrogênio e favorece a estabilidade da estrutura. Em posse dessas caracterizações, foi possível realizar cálculos preliminares da aplicação de um potencial externo sobre o sistema por meio formalismo de Funções de Green Fora do Equilíbrio (NEGF).

 \vspace{\onelineskip}
    
 \noindent
 \textbf{Palavras-chaves}:  DFT, interfaces heterogêneas, eletroquímica
\end{resumo}