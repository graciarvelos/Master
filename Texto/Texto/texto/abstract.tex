\begin{resumo}[Abstract]
 \begin{otherlanguage*}{english}
New ways of generating clean energy are related to reactions that occur at water/metal interfaces, for example via electrochemical and heterogeneous catalysis processes. Therefore, the characterization at the atomic level of these processes is fundamental for the improvement of the current ones and the development of new renewable energy sources. Thus, it is necessary to understand the structural and electronic properties of water/metal interfaces, as well as the system's responses to external perturbations. Therefore, in this work Density Functional Theory (DFT) was used to characterize the interactions of water (monomer and layer) on metallic surfaces (palladium and gold). The Non-Equilibrium Green's Function (NEGF) coupled to DFT was used in order to properly compute an external bias potential on the water/metal interface and to investigate how it affects the structural, electronic, and vibrational properties. Thus, we observe that the molecules tend to approach the metal when it is negatively charged and move away when it is positively charged. These changes affect the vibrational properties and are strongly dependent on the reactivity of the metal.




   \vspace{\onelineskip}
 
   \noindent 
   \textbf{Key-words}: DFT, heterogeneous interface, electrochemistry
   

 \end{otherlanguage*}
\end{resumo}